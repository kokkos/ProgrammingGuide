\section{Related Work}\label{chap:related}
\subsection{RAJA}
Like Kokkos, RAJA is a C++ framework for porting massive DOE codes to a variety of architectures. In both frameworks, a developer first expresses the operations their kernel performs using simple C++ functional programming mechanisms. The difference is that Kokkos is declarative, while RAJA is imperative. In Kokkos, the framework figures out how to map that kernel to different architectures. In RAJA, functionality is provided to make it easy for a developer to try different strategies to map those kernels to architectures, but the developer still decides how that mapping happens. Which model makes more sense in a given circumstance depends on how much exposure to the underlying programming model a developer wants. RAJA would be ideal for a class in which the educator wants to expose students to the semantics of different programming models without exposing them to the detailed syntax of those models.
\subsection{DPC++}
DPC++ is a model developed by Intel(TM) for expressing parallelism across their architectures. In addition to sharing a declarative philosophy with Kokkos, it even share a number of patterns. Unlike Kokkos, at time of publication it targets only Intel architectures (it isn't vendor-agnostic) For a course targeting Intel(TM) platforms, you're likely to see good tool support, DPC++ could be a valid option.
\subsection{Parallel Constructs in the C++ Standard}
C++ increasingly supports parallel constructs. We discuss the contributions of Kokkos to C++ in~\ref{chap:c++}, it's increasingly possible to teach parallelism with nothing but a C++ compiler. A problem is that at time of publication C++ doesn't support accelerators, or many of the complexities of parallel computing as it's used in practice. A positive is that instead of requiring libraries, you just need a modern C++ compiler. For a course in which topics like accelerators aren't discussed, this can be a valid option.
%\end{itemize}

