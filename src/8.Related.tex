\section{Related Work}\label{chap:related}
Like Kokkos, RAJA~\cite{RAJA} is a C++ framework for porting massive DOE codes to a variety of architectures. In both frameworks, a developer first expresses the operations their kernel performs using simple C++ functional programming mechanisms. The difference is that Kokkos is declarative, while RAJA is descriptive. In Kokkos, the framework figures out how to map that kernel to different architectures. In RAJA, functionality is provided to make it easy for a developer to try different strategies to map those kernels to architectures, but the developer still decides how that mapping happens. Which model makes more sense in a given circumstance depends on how much exposure to the underlying programming model a developer wants. RAJA is a viable option for class education in which the educator wants to expose students to the semantics of different programming models without exposing them to the detailed syntax of those models.

DPC++\cite{DPCPP} is a model developed by Intel\textsuperscript{\textregistered} for expressing parallelism across Intel architectures. In addition to sharing a descriptive philosophy with Kokkos, it also shares parallel patterns. Unlike Kokkos, at time of publication it targets only Intel architectures and is not vendor-agnostic. A course targeting Intel architectures, DPC++ could be a valid option.

C++ increasingly supports parallel constructs\cite{CPP}. We have highlighted contributions of Kokkos to C++ in~\ref{chap:c++}. It is becoming increasingly possible to teach parallelism in C++ and relying on modern C++ compilers. However, at time of publication, C++ does not support accelerators, or many of the complexities of parallel computing related to accelerator programming. In courses that do not target accelerators, this can be a relevant option.
 
% Alpaka Dresden
% NV Thrust
