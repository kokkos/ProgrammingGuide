\section{Related Work}\label{chap:related}
RAJA~\cite{RAJA} is a C++ abstraction library with similarities to Kokkos. The important different however is that Kokkos is descriptive, while RAJA is prescriptive. In Kokkos, the framework determines how an application is mapped to the underlying hardware. RAJA provides functionality that exposes hardware details and relies on the developer to try different strategies to map kernels to architectures. RAJA is a viable option for class education in which the educator would like to expose students to the semantics of different programming models without exposing them to the detailed syntax of those models.

DPC++\cite{DPCPP} is a model developed by Intel\textsuperscript{\textregistered} for expressing parallelism across Intel architectures. In addition to sharing a descriptive philosophy with Kokkos, it also shares parallel patterns. Unlike Kokkos, at time of publication it targets only Intel architectures and is not vendor-agnostic. A course targeting Intel architectures, DPC++ could be a valid option.

C++ increasingly supports parallel constructs. We have highlighted contributions of Kokkos to C++ in~\ref{chap:c++}. It is becoming increasingly possible to teach parallelism in C++ and relying on modern C++ compilers. However, at time of publication, C++ does not support accelerators, or many of the complexities of parallel computing related to accelerator programming. In courses that do not target accelerators, this can be a relevant option.
 
% Alpaka Dresden
% NV Thrust
