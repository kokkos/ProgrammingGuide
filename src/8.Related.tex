\section{Related Work}\label{chap:related}


C++ increasingly supports parallel constructs. It is becoming increasingly possible to teach parallelism in C++ and relying on modern C++ compilers. However, at the time of publication, C++ does not support accelerators or many of the complexities of parallel computing related to accelerator programming. In courses that do not target accelerators, this can be a relevant option.

Over the past decade the Kokkos programming model has contributed to a number of current and near-future ISO-C++ features, including \emph{atomic\_ref}\cite{wg21_p0019}, \emph{mdspan}\cite{MDSPAN}, and executors.  In particular, the design of C++ executors, which are similar to Kokkos execution spaces, were influenced by this programming model. As the demand to extract performance from increasingly deep and increasingly asynchronous software stacks across a wide variety of domains increases, many C++ experts expect executors to become a central abstraction in any performance-oriented software stack. 

RAJA~\cite{RAJA} is a C++ abstraction library with similarities to Kokkos. The important difference however is that Kokkos is descriptive, while RAJA is prescriptive. In Kokkos, the programming models determine how an application is mapped to the underlying hardware. RAJA provides functionality that exposes hardware details and relies on the developer to try different strategies to map kernels to architectures. RAJA is a viable option for class education in which the educator would like to expose students to the semantics of different programming models without exposing them to the detailed syntax of those models.

DPC++\cite{DPCPP} is a model developed by Intel\textsuperscript{\textregistered} for expressing parallelism across Intel architectures. In addition to sharing a descriptive philosophy with Kokkos, it also shares parallel patterns. Unlike Kokkos, it targets Intel architectures only and is not vendor-agnostic. A course targeting Intel architectures, DPC++ could be a valid option.
 
% Alpaka Dresden
% NV Thrust
