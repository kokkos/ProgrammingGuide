
\section{Introduction}\label{chap:introduction}

Parallel and distributed computing is a focus at universities and teaching institutions. The interest is two-fold. Firstly, given their role, educators adapt to trends in research and industry and include parallel computing in their curricula. Secondly, parallel computing is an attractive field for department research and industry collaborations. At Sandia, we develop a performance portability library called Kokkos that is ideal for educators looking to teach parallel computing.

In teaching a class on parallel programming, educators must balance considerations of architecture, programming model, and theoretical concepts. Unfortunately, not all architectures support all programming models, nor do many programming models cleanly express all parallel programming constructs. Many vendor models compound this, requiring students to have access to specific hardware, and limiting the concepts of parallelism to those supported by that hardware. This makes the choice of programming model important and difficult.

Consequently, generic parallel programming models are needed that target mainstream programming languages and allow a vendor- and architecture independent, performance-portable expression of concurrency. In this way, a programming exercise for students becomes capable of running efficiently on parallel computer architectures that conform to a common parallel execution model. This compares to the support of Von Neumann architectures in programming languages today. The lack of such parallel programming models motivated us to develop Kokkos.

Kokkos\footnote{from Greek for grain or seed}~\cite{KOKKOS_PAPER_HERE}, is a performance-portable, parallel programming model for C++ developed at Sandia National Laboratories (SNL) for developing million line scientific software on the world's biggest supercomputers. It consists of a programming model specification and a library. The specification defines an execution and memory model and an application programming interface (library API). The runtime library exposes functions representing parallel patterns and types for data management and maps them to the underlying abstract execution model. The Kokkos library maps those patterns to vendor-specific programming models, libraries, and memory layouts. Kokkos has been successfully used in many large-scale projects at SNL, LLNL and other research centers~\cite{CITEKOKKOSUSECASES}. Its key properties are

\begin{itemize}
\item Parallel programming model for C++
\item Targets an abstract execution model
\item Implements parallel patterns and tasking paradigms
\item Offers abstract data types for performance portability
\item Pure C++ implementation through template metaprogramming
\item Aligns with C++ standardization efforts for parallel programming
\item Vendor- and architecture independent and open-source
\item Includes exercises and teaching material
\end{itemize}

The objective of this paper is to present Kokkos from two perspectives. Firstly, we would like to present our thoughts that accompanied the design and development of the Kokkos parallel programming model. These insights may be considered useful in parallel programming classes that introduce programming models and reason about design choices.

Further, we would like to present Kokkos to the educational community as one representative parallel programming model for a trend towards architecture-independent and performance-portable programming of CPUs and accelerators. We show an abstraction with enough semantic information to produce performance-portable, parallel code on current architectures. We highlight its alignment and contributions to standardization efforts towards supporting parallel programming in C++. 

Secondly, we would like to promote Kokkos to students and educators as a platform to understand and experiment with parallel patterns in a fashion that is

\begin{itemize}
\item Generic
\item Vendor-agnostic
\item Declarative
\end{itemize}

Kokkos has semantics that are generic. This means that students are learning parallelism detached from any architectural concerns. Kokkos won't require them to know the specifics of the "symmetric multiprocessors" of a GPU or the cache hiearchy on a CPU\footnote{Though nothing precludes an educator from teaching these independently of the model}. Kokkos is vendor agnostic. If you have a computer lab with processors from one vendor and your students have processors from another, this presents no problem, Kokkos likely works on both. Kokkos is declarative, students aren't saying "map these work items to this set of threads" but "do a reduction over this set of data." This is critical, much of software engineering is moving to declarative models and languages, learning to operate in such a mindset is useful outside of parallel programming. All of these are desirable in a programming model for education.

The rest of this paper is structured as follows. The next Chapter gives an overview on parallel programming concepts and presents the notion of \emph{semantic capture}. Chapter~\ref{chap:kokkosMM} presents the abstract machine model used to derive required semantic information for performance portability in Kokkos. Chapters~\ref{chap:kokkosPM} and ~\ref{chap:kokkosBackend} present the Kokkos programming model and its back-end support. Chapters~\ref{chap:c++} and~\ref{chap:related} show the alignment to C++ standardization efforts and dicuss related work and lastly Chapter~\ref{chap:conclusion} concludes this work.
