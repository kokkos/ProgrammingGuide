\section{Virtual Teaching Experiences}\label{chap:teaching}

In our experience, teaching Kokkos benefits from a set of good practices that we document in this section. We exclude obvious practices such as good documentation, accessibility, or responsiveness to user questions. Our best practices can be divided into accessibility and organization of source code distribution, presentation material, and presentation methodology. 

Our Kokkos source code is accessible through GitHub and supports a variety of build systems including Makefile, Cmake as well as Spack\cite{SPACK}. This variety in build system support makes the code easily accessible and executable to a wider range of users who are familiar with one of these build systems. Further, the Kokkos source code directly includes an example for each feature. The inclusion of examples in the source code distribution is beneficial for quick look-up of syntax, APIs, and usage patterns. All examples are embedded into small but complete test applications as this unburdens the interested programmer from any additional programming. For completeness, examples include codes that illustrate important performance implications of certain properties such as the definition of access patterns in views that differ between host and accelerator devices (strided versus coalesced). 

Our presentation and tutorial material is divided into 8 units that gradually roll out all features and use-cases of the Kokkos programming model. Our tutorial material covers approximately 8 two-hour lectures, where each lecture summarizes the key learning points of the previous lecture at the beginning, summarizes the current lecture at the end, and gives a preview on the upcoming lecture. This creates continuity, improves the attendee's orientation in the subject, and spurs interest. Each lecture is concluded with a call for experimentation. Each lecture has a prepared hands-on exercise that can be used either in-lined during presential lectures or given out as homework for individual experimentation. During online lectures, the presenter explains the solution of the homework through screen-sharing and repeats important concepts. We would like to point out that each homework assignment contains the solution. Providing a solution lowers the barrier of experimentation and helps impatient students.

Online lectures tend to restrict possible interaction between the presenter and attendees to chats. In turns out useful to assign assistant presenters to actively participate in the chat during the presentation. This helps to resolve questions and support audience engagement. Since some chat conversations tend to fall short, it is a good practice to save the chat transcript and to publish the transcript together with refined responses on-line where they can be accessed by attendees for reference. To help interested attendees, we offer consulting hours. Consulting hours are hosted as online telephone conferences where attendees can call in and ask further questions regarding lecture content or homework.

The advantage of virtual classes is their possibility of being recording. Recordings can serve as a valuable resource for future reference with quick access links pointing to particular moments in the recording relevant to a topic. 

A recent online course\cite{KOKKOS_LECTURE} has been publicly announced through social media and counted approximately 200 attendees. The majority of attendees are familiar with at least one parallel programming model and have a background in scientific computing. The listing in Figure~\ref{fig:lectureOutline} shows the structure of our on-line lecture.

Question asked by attendees during the recent online course showed that they did understand the relevance as well as key concepts. Questions were mostly related to details in behavior, differences to other programming models and best practices. We would like to direct the interested reader to the our online website where the Q\&A sessions are listed in detail.

\begin{figure}

\begin{itemize}
\item Introduction, Building and Parallel Dispatch
\item Views and Spaces
\item Data Structures + MultiDimensional Loops
\item Hierarchical Parallelism
\item Tasking, Streams and SIMD
\item Internode: MPI and PGAS
\item Tools: Profiling, Tuning and Debugging
\item Kernels: Sparse and Dense Linear Algebra
\end{itemize}
\caption{Outline of the Kokkos on-line lecture series with 8 two-hour sessions incrementally cover all concepts and features.}
\label{fig:lectureOutline}
\end{figure}




