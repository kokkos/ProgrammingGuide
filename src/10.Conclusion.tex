\section{Conclusion}\label{chap:conclusion}
In this work, we gave an introduction to the Kokkos parallel programming model and presented guiding thoughts behind its design. We discussed a machine representation of a parallel hardware architecture that is characterized by a variety of processing units, memories, and their hierarchical organization. This machine model allowed us to define a constrained semantic represented through common abstractions such as patterns, views, and spaces. 

An example application gave insight into parallel programming with Kokkos and its OpenMP backend implementation. We used them to show the generic programming interface and type specialization of the implementing classes to support a particular vendor programming model.

To the educational community, we presented Kokkos as a representative programming model of a trend towards the support of parallel programming in base languages and libraries through a common vendor- and architecture-independent abstractions.

The recent shift towards online classes impacted our teaching of Kokkos as well. In this work, we presented our best-practices when conducting online classes including lecture structure, material preparation, and methodology. 

Finally, we would like to express gratitude to educators of parallel programming and computer science for taking on this important and challenging role. In case of interest, we would gladly provide further material and support on Kokkos and offer to the interested reader to reach out to the development team on Slack\cite{KOKKOS_SLACK}.
