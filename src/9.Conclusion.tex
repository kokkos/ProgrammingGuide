\section{Conclusion}\label{chap:conclusion}
In this work we gave an introduction to the Kokkos parallel programming model and presented guiding thoughts behind its design. We discussed a machine representation of a parallel hardware architecture that is characterized by a variety of processing units, memories and their hierarchical organization. This machine model allowed us to define a constrained semantic represented though different types and patterns. Further we listed supported paradigms and considerations that influenced the design of the semantic capture.

An example application gave insight into parallel programming with patterns and into the back-end implementation. We used them to show the generic programming interface and type specialization of the implementing classes to support a particular vendor programming model.

To the educational community, we presented Kokkos as a representative programming model of a trend towards support of parallel programming in base languages and libraries. We believe in the importance of embracing this trend in computer science education as such parallel programming models can contribute to the generic and relevant education of parallel programming and prepare student for their programming careers.

Finally we would like to express gratitude to educators for taking on this important role. We are open to provide further material or support and propose the interested reader to reach out to the development team on Slack\cite{KOKKOS_SLACK}.
