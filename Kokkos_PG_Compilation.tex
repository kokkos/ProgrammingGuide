
\chapter{Compiling}\label{C:build}

This chapter explains how to compile Kokkos, and how to link your
application against Kokkos.  Kokkos supports two build systems:
\begin{itemize}
\item Using the embedded Makefile
\item Trilinos' CMake build system
\end{itemize}
Note that the two explicitly supported build methods should not be
mixed.  For example, do not include the embedded Makefile in your
application build process, while explicitly linking against a
pre-compiled Kokkos library in Trilinos.  We also include specific
advice for building for NVIDIA GPUs, and for Intel Xeon Phi.

\section{General Information}\label{S:build:gen}

Kokkos consists mainly of header files. 
Only a few functions have to be compiled into object files outside of the application's source code.
Those functions are contained in \verb!.cpp! files inside the \lstinline|kokkos/core/src| directory and its subdirectories.
The files are internally protected with macros to prevent compilation if the related execution space is not enabled. 
Thus, it is not necessary to create a list of included object files specific to your compilation target.
One may simply compile all \verb!.cpp! files. 
The enabled features are controlled via macros which have to be provided in the compilation line or in the \lstinline|KokkosCore_config.h| include file. 
A list of macros can be found in Table \ref{TBL:CompileMacros}.
\begin{table}
\caption{Table of configuration Macros}
\label{TBL:CompileMacros}
\begin{small}
\begin{tabular}[t]{p{0.25\textwidth}p{0.3\textwidth}p{0.42\textwidth}}
\hline\hline
Macro & Effect & Comment \\
\hline
{\tiny KOKKOS\_HAVE\_CUDA} & Enable the CUDA execution space. & Requires a compiler capable of understanding CUDA-C. See Section \ref{S:build:CUDA}. \\
{\tiny KOKKOS\_HAVE\_OPENMP} & Enable the OpenMP execution space. & Requires the compiler to support OpenMP (e.g., \verb!-fopenmp!). \\
{\tiny KOKKOS\_HAVE\_PTHREADS} & Enable the Threads execution space. & Requires linking with libpthread.\\
{\tiny KOKKOS\_HAVE\_CXX11} & Enable internal usage of C++11 features. & The code needs to be compile with the C++11 standard. Most compilers accept the \verb!-std=c++11! flag for this.\\
{\tiny KOKKOS\_HAVE\_HWLOC} & Enable thread and memory pinning via hwloc. & Requires linking with libhwloc.\\
{\tiny KOKKOS\_DONT\_INCLUDE \_KOKKOSCORE\_CONFIG\_H} & Do not include the KokkosCore\_config.h file. & Useful when providing all macros via the compile line instead of auto-generating that file. 
This option is set for the embedded Makefile. \\
\hline\hline
\end{tabular}
\end{small}
\end{table}

\section{Using Kokkos' Makefile system}\label{S:build:make}

Kokkos provides an embedded Makefile (\lstinline|kokkos/Makefile.kokkos|) for inclusion in an application Makefile. 
This embedded Makefile generates a \lstinline|KOKKOS_INC| and a \newline
\lstinline|KOKKOS_LINK| variable which need to be appended to the compile and the link line respectively in the application Makefile. 
If you want to include dependencies (i.e. trigger a rebuild of the application object files when Kokkos files change) you can include \lstinline|KOKKOS_HEADERS| in you dependency list.
Note that you cannot compile and link at the same time!
Makefile.kokkos uses a number of variables which can be set either in the parent Makefile prior to including it, or on the command line.
These variables take the form of \lstinline|SOME_VAR=yes/no| or \lstinline|SOME_PATH=/path/to/library| and control build options such as the target architecture. 
A list can be found in Table \ref{TBL:MakefileOptions}.
Example application Makefiles can be found in the tutorial examples under \verb!kokkos/example/tutorial!.


\begin{table}
\caption{Table of Makefile options}
\label{TBL:MakefileOptions}
\begin{small}
\begin{tabular}[t]{lp{0.7\textwidth}}
\hline\hline
Option & Description \\
\hline
CUDA & Enable the CUDA execution space. \\
OMP & Enable the OpenMP execution space. \\
PTHREADS & Enable the Threads execution space.\\
MIC & Enable compilation for Intel Xeon Phi. \\
CXX11 & Enable internal usage of C++11 features.\\
HWLOC & Enable hwloc usage.\\
LIBRT & Enable librt usage for the \lstinline|Kokkos::Impl::Timer| class.\\
CUDA\_PATH & Set the path to the CUDA toolkit directory.\\
HWLOC\_PATH & Set the path to the hwloc library.\\
\hline\hline
\end{tabular}
\end{small}
\end{table}

\section{Using Trilinos' CMake build system}\label{S:build:Trilinos}

The Trilinos project (see \url{trilinos.org}) is an effort to develop
algorithms and enabling technologies within an object-oriented
software framework for the solution of large-scale, complex
multiphysics engineering and scientific problems.  Trilinos is
organized into packages.  Even though Kokkos is a stand-alone software
project, Trilinos uses Kokkos extensively.  Thus, Trilinos' source
code includes Kokkos' source code, and builds Kokkos as part of its
build process.

Trilinos' build system uses CMake.  Thus, in order to build Kokkos as
part of Trilinos, you must first install CMake (version
\texttt{2.8.12} or newer; CMake \texttt{3.x} works).  
To enable Kokkos when building Trilinos, set the CMake option \verb!Trilinos_ENABLE_Kokkos!.
Trilinos' build system lets packages express dependencies on other packages or external libraries.
If you enable any Trilinos package (e.g., Tpetra) that has a required dependency on Kokkos, 
Trilinos will enable Kokkos automatically.
Configuration macros are automatically inferred from Trilinos settings. 
For example, if the CMake option \lstinline|Trilinos_ENABLE_OpenMP| is ON, Trilinos will define the macro \lstinline|KOKKOS_HAVE_OPENMP|.
Trilinos' build system will autogenerate the previously mentioned \lstinline|KokkosCore_config.h| file that contains those macros. 

We refer readers to Trilinos' documentation for details.  Also, the
\texttt{kokkos/config} directory includes examples of Trilinos
configuration scripts.
 
\section{Building for CUDA}\label{S:build:CUDA}

Any Kokkos application compiled for CUDA embeds CUDA code via template metaprogramming.
Thus, the whole application must be built with a CUDA-capable compiler.
(At the moment, the only such compiler is NVIDIA's NVCC.)
More precisely, every compilation unit containing a Kokkos kernel or a function called from a Kokkos kernel has to be compiled with a CUDA-capable compiler. 
This includes files containing \lstinline|Kokkos::View| allocations, which call an initialization kernel. 

The current version of NVCC has some shortcomings when used as the main compiler for a project, in particular when part of a complex build system.
For example, it does not understand most GCC command-line options, which must be prepended by the \lstinline|-Xcompiler| flag when calling NVCC. 
Kokkos comes with a shell script, called \lstinline|nvcc_wrapper|, that wraps NVCC to address these issues.
We intend this as a drop-in replacement for a normal GCC-compatible compiler (e.g., GCC or Intel) in your build system.
It analyzes the provided command-line options and prepends them correctly. 
It also adds the correct flags for compiling generic C++ files containing CUDA code (e.g., \verb!*.cpp!, \verb!*.cxx!, or \verb!*.CC!).
By default \lstinline|nvcc_wrapper| calls \verb!g++! as the host compiler.
You may override this by providing NVCC's '\lstinline|-ccbin|' option as a compiler flag.
The default can be reset by editing the script itself. 

Many people use a system like Environment Modules (see \\ \url{http://modules.sourceforge.net/}) to manage their shell environment.
When using a module system, it can be useful to provide different versions for different back-end compiler types (e.g., \verb!icpc!, \verb!pgc++!, \verb!g++!, and \verb!clang!).
To use the \lstinline|nvcc_wrapper| in conjunction with MPI wrappers, simply overwrite which C++ compiler is called by the MPI wrapper. 
For example, you can reset OpenMPI's C++ compiler by setting the \lstinline|OMPI_CXX| environment variable.
Make sure that \lstinline|nvcc_wrapper| calls the host compiler with which the MPI library was compiled.

\section{Building for Intel Xeon Phi}\label{S:build:Phi}

TODO