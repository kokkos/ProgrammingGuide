\documentclass[a4paper,11pt]{book}
\usepackage[utf8x]{inputenc}

\usepackage{graphicx}

%% or use the epsfig package if you prefer to use the old commands
%% \usepackage{epsfig}

%% The amssymb package provides various useful mathematical symbols
%% \usepackage{amssymb}
%% The amsthm package provides extended theorem environments
%% \usepackage{amsthm}

%% The lineno packages adds line numbers. Start line numbering with
%% \begin{linenumbers}, end it with \end{linenumbers}. Or switch it on
%% for the whole article with \linenumbers after \end{frontmatter}.
%% \usepackage{lineno}

%% \usepackage{placeins}

\usepackage{listings}
\usepackage{subfigure}
\usepackage{courier}
\usepackage{tikz}
\usetikzlibrary{shapes,positioning}
\usepackage{hyperref}
\hypersetup{
    colorlinks,
    citecolor=black,
    filecolor=black,
    linkcolor=black,
    urlcolor=black
}
\usepackage{url}

\usepackage{color}

\definecolor{lightlightgray}{gray}{0.97}
\definecolor{lightlightgreen}{rgb}{0.8,1,0.8}
\definecolor{darkblue}{rgb}{0,0,0.8}
\definecolor{darkgreen}{rgb}{0,0.5,0}
\definecolor{LightSkyBlue}{rgb}{0.8,1,0.8}
\definecolor{MidnightBlue}{rgb}{0.8,1,0.8}
\definecolor{SandiaGray}{RGB}{130, 120, 111}
\definecolor{SandiaBlue}{RGB}{0, 50, 90}
\definecolor{SandiaLightBlue}{RGB}{0,150,180}
\definecolor{SandiaLightLightBlue}{RGB}{0,200,220}
\definecolor{SandiaLightLightLightBlue}{RGB}{0,228,238}
\definecolor{SandiaRed}{RGB}{130, 36, 51}
\lstset{language=C++,
           frame=single,
           basicstyle=\ttfamily\scriptsize,
           %%basicstyle=\ttfamily,
           %%basicstyle=\sffamily\normalsize,
           keywordstyle=\color{darkblue},
           backgroundcolor=\color{lightlightgray},
           commentstyle=\color{darkgreen}
           }

\tikzset{HighLightBox/.style = {draw=SandiaBlue, fill=SandiaLightLightLightBlue, line width=2pt,
                          rectangle, rounded corners, inner sep=10pt, inner ysep=20pt},
         fancytitle/.style ={fill=SandiaBlue, text=white, rounded corners}
         }



\begin{document}
\begin{minipage}[h]{\paperwidth}
  {\begin{tikzpicture}[remember picture,overlay]
    \node[] at (current page.north west)
      {\begin{tikzpicture}[remember picture, overlay, scale=0.5]
          \draw[fill=SandiaLightLightBlue,color=SandiaLightLightBlue] (0,-10.5cm) rectangle (8cm,5.5cm);
          \draw[fill=SandiaLightBlue,color=SandiaLightBlue] (0,-8.5cm) rectangle (6cm,3.5cm);
          \draw[fill=SandiaBlue,color=SandiaBlue] (0,-6.5cm) rectangle (4cm,1.5cm);          
          \draw[fill=SandiaLightBlue,color=SandiaLightBlue] (4cm,-6.5cm) circle [radius=4cm];
          \draw[fill=SandiaLightLightBlue,color=SandiaLightLightBlue] (6cm,-8.5cm) circle [radius=6cm];
          \draw[fill=white,color=white] (8cm,-10.5cm) circle [radius=8cm];
          \draw[fill=SandiaGray,color=SandiaGray] (0,0cm) rectangle (2\paperwidth,-0.5cm);
          \draw[fill=SandiaBlue,color=SandiaBlue] (0,-0.5cm) rectangle (2\paperwidth,-2.5cm);
       \end{tikzpicture}
      };
   \end{tikzpicture}
   }

  {\begin{tikzpicture}[remember picture,overlay]
      \node[] at (current page.south east)
      {\begin{tikzpicture}[remember picture, overlay,scale=0.5]
          \draw[fill=SandiaLightLightBlue,color=SandiaLightLightBlue] (-8cm,11.5cm) rectangle (8cm,5.5cm);
          \draw[fill=SandiaLightBlue,color=SandiaLightBlue] (-6cm,9.5cm) rectangle (6cm,3.5cm);
          \draw[fill=SandiaBlue,color=SandiaBlue] (0,7.5cm) rectangle (-4cm,1.5cm);          
          \draw[fill=SandiaLightBlue,color=SandiaLightBlue] (-4cm,7.5cm) circle [radius=4cm];
          \draw[fill=SandiaLightLightBlue,color=SandiaLightLightBlue] (-6cm,9.3cm) circle [radius=6cm];
          \draw[fill=white,color=white] (-8cm,11.5cm) circle [radius=8cm];
        \draw[fill=SandiaBlue,color=SandiaBlue] (-2\paperwidth,0.5cm) rectangle
          (\paperwidth,3.5cm);
        \draw[fill=SandiaGray,color=SandiaGray] (-2\paperwidth,0) rectangle
          (\paperwidth,0.5cm);
       \end{tikzpicture}
      };                           
   \end{tikzpicture}
   }

  {\begin{tikzpicture}[remember picture,overlay]
     \node[HighLightBox,yshift=-7cm,xshift=0.25\paperwidth] at (current page.north west) (ParallelPatterns)
     {\begin{minipage}[h]{0.4\paperwidth}
Algorithms are expressed in parallel patterns:
\begin{lstlisting}
\\ Execute Functor Operator 
parallel_for( ExecutionPolicy , Functor);
\end{lstlisting}
\begin{lstlisting}
\\ Execute Functor Operator and perform Reduction 
parallel_reduce( ExecutionPolicy , Functor);
parallel_reduce( ExecutionPolicy , Functor, Result);
\end{lstlisting}
\begin{lstlisting}
\\ Execute Functor Operator and perform Scan 
parallel_scan( ExecutionPolicy , Functor);
\end{lstlisting}
        
      \end{minipage}
      };
      \node[fancytitle, right=10pt] at (ParallelPatterns.north west) {Parallel Patterns};   
      \end{tikzpicture}
  }
  {\begin{tikzpicture}[remember picture,overlay]
     \node[HighLightBox,yshift=-15cm,xshift=0.25\paperwidth] at (current page.north west) (ExecutionPolicies)
     {\begin{minipage}[h]{0.4\paperwidth}
Algorithms are expressed in parallel patterns:
\begin{lstlisting}
\\ Execute Functor Operator 
parallel_for( ExecutionPolicy , Functor);
\end{lstlisting}
\begin{lstlisting}
\\ Execute Functor Operator and perform Reduction 
parallel_reduce( ExecutionPolicy , Functor);
parallel_reduce( ExecutionPolicy , Functor, Result);
\end{lstlisting}
\begin{lstlisting}
\\ Execute Functor Operator and perform Scan 
parallel_scan( ExecutionPolicy , Functor);
\end{lstlisting}
        
      \end{minipage}
      };
      \node[fancytitle, right=10pt] at (ExecutionPolicies.north west) {Execution Policies};   
      \end{tikzpicture}
  }
\end{minipage}


\end{document}
